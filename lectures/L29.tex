\include{header}

\begin{document}

\lecture{29 --- Asynchronous I/O}{\term}{Jeff Zarnett, based on original by Patrick Lam}

\section*{Asynchronous (non-blocking) I/O}

\begin{center}
   ``Asynchronous I/O, for example, is often infuriating.''\\
\hfill --- Robert Love. {\em Linux System Programming, 2nd ed, } page 215.

\end{center}

To motivate the need for non-blocking I/O, consider some of the usual I/O code:

\begin{lstlisting}[language=C]
    fd = open(...);
    read(...);
    close(fd);
\end{lstlisting}

As we discussed much earlier, the \texttt{read} call is blocking, as expected. So, your program waits for the I/O operation to be complete before continuing on to the next statements (whatever they are). This is sometimes, but not always, sensible. 

If you are waiting for the bus, do you stare off blankly into space while waiting for it to arrive? Probably not. More likely you pull out your phone and start to use it for something. Whether that is productive or not (e.g., answering a project e-mail or liking posts on Facebook) is up to you, but you are doing something and making use of the time. 

Our main solution until now is threads: if one thread gets blocked on the I/O the other ones can continue and is fine. But maybe you don't want to use threads, or maybe you can't due to: race conditions, thread stack size overhead, or limitations on the maximum number of threads. The last one might seem ridiculous but in some embedded system you may not have the option to make new threads. 

Sometimes, also, your programming language (e.g., Javascript) doesn't allow you to make multiple threads and you really have no choice but to use asynchronous I/O. It's a useful tool to have in the toolbox so let's get into it.

The simplest example:

\begin{lstlisting}[language=C]
    fd = open(..., O_NONBLOCK);
    read(...); // returns instantly!
    close(fd);
\end{lstlisting}

In principle, the {\tt read} call is supposed to return instantly,
whether or not results are ready. That was easy! Well, not so much. The {\tt O\_NONBLOCK} flag actually only has the desired behaviour on sockets. The semantics of {\tt O\_NONBLOCK} is for I/O calls to not block, in the sense that they should never wait for data while there is no data available. But unfortunately, files \textit{always} have data available. 

\paragraph{Conceptual view: non-blocking I/O.} Fundamentally,
there are two ways to find out whether I/O is ready to be queried:
polling (under UNIX, implemented via {\tt select}, {\tt poll},
and {\tt epoll}) and interrupts (under UNIX, signals). Under Linux, there are  {\tt aio} calls to be able to send requests to the I/O subsystem asynchronously and not, for instance, wait for the disk to spin up and deliver the data to you. But that's implementation specific. 


The key idea is to give {\tt epoll} a bunch of file descriptors and
wait for events to happen. In particular:
     \begin{itemize}
       \item create an epoll instance ({\tt epoll\_create1});
       \item populate it with file descriptors ({\tt epoll\_ctl}); and
       \item wait for events ({\tt epoll\_wait}).
     \end{itemize}
Let's run through these steps in order.

\paragraph{Creating an {\tt epoll} instance.} Just use the API:
    \begin{lstlisting}[language=C]
   int epfd = epoll_create1(0);
    \end{lstlisting}

The return value {\tt epfd} is typed like a UNIX file
descriptor---{\tt int}---but doesn't represent any files; instead, use
it as an identifier, to talk to {\tt epoll}.

The parameter ``{\tt 0}'' represents the flags, but the only available flag
is {\tt EPOLL\_CLOEXEC}. Not interesting to you.

\paragraph{Populating the {\tt epoll} instance.} Next, you'll want
{\tt epfd} to do something. The obvious thing is to add some {\tt fd}
to the set of descriptors watched by {\tt epfd}:
    \begin{lstlisting}[language=C]
   struct epoll_event event;
   int ret;
   event.data.fd = fd;
   event.events = EPOLLIN | EPOLLOUT;
   ret = epoll_ctl(epfd, EPOLL_CTL_ADD, fd, &event);
    \end{lstlisting}

You can also use {\tt epoll\_ctl} to modify and delete descriptors from {\tt epfd}; read the manpage to find out how.

\paragraph{Waiting on an {\tt epoll} instance.} Having completed
the setup, we're ready to wait for events on any file descriptor in {\tt epfd}.
    \begin{lstlisting}[language=C]
  #define MAX_EVENTS 64

  struct epoll_event events[MAX_EVENTS];
  int nr_events;

  nr_events = epoll_wait(epfd, events, MAX_EVENTS, -1);
    \end{lstlisting}

The given {\tt -1} parameter means to wait potentially forever;
otherwise, the parameter indicates the number of milliseconds to wait.
(It is therefore ``easy'' to sleep for some number of milliseconds by
starting an {\tt epfd} and using {\tt epoll\_wait}; takes two function
calls instead of one, but allows sub-second latency.)

Upon return from {\tt epoll\_wait}, we know that we have {\tt
  nr\_events} events ready.

\subsection*{Level-Triggered and Edge-Triggered Events}
One relevant concept for these polling APIs is the concept of
\emph{level-triggered} versus \emph{edge-triggered}.  The default {\tt
  epoll} behaviour is level-triggered: it returns whenever data is
ready. One can also specify (via {\tt epoll\_ctl}) edge-triggered
behaviour: return whenever there is a change in readiness.

%\paragraph{Another Live Demo.} Now let's run some code (socket.c) that creates a
%server and reads from it, in either level-triggered mode or edge-triggered mode.

One would think that level-triggered mode would return from {\tt read}
whenever data was available, while edge-triggered mode would return
from {\tt read} whenever new data came in. Level-triggered does behave
as one would guess: if there is data available, {\tt read()} returns
the data. However, edge-triggered mode returns whenever the
state-of-readiness of the socket changes (from no-data-available to
data-available). Play with it and get a sense for how it works.

Good question to think about: when is it appropriate to choose one or the other?

\subsection*{Asynchronous I/O}
As mentioned above, the POSIX standard defines {\tt aio} calls.
Unlike just giving the {\tt O\_NONBLOCK} flag, using {\tt aio} works
for disk as well as sockets.

\paragraph{Key idea.} You specify the action to occur when I/O is ready:
    \begin{itemize}
      \item nothing;
      \item start a new thread; or
      \item raise a signal.
    \end{itemize}

Your code submits the requests using e.g. {\tt aio\_read} and {\tt aio\_write}.
If needed, wait for I/O to happen using {\tt aio\_suspend}.



\bibliographystyle{alphaurl}
\bibliography{252}


\end{document}
