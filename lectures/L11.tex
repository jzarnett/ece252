\input{header.tex}

\begin{document}

\lecture{ 11 --- Threads Continued}{\term}{Jeff Zarnett}

\section*{POSIX Threads}

Let's start with a different example:

\begin{lstlisting}[language=C]
#include <pthread.h>
#include <stdio.h>

int sum; /* this data is shared by the thread(s) */
void *runner(void *param); /* threads call this function */
int main(int argc, char *argv[]) {

  pthread_t ti; /* the thread identifier */
  pthread_attr_t attr; /* set of thread attributes */

  if (argc != 2) {
    fprintf(stderr,"usage: a.out <integer value>\n"); 
    return -1;
  }
  if (atoi(argv[1]) < 0) {
    fprintf(stderr, "%d must be >= 0\n", atoi(argv[1])); 
    return -1;
  }

  /* get the default attributes */
  pthread_attr_init(&attr);
  /* create the thread */
  pthread_create(&tid, &attr, runner, argv[1]); /* wait for the thread to exit */
  pthread_join(tid, NULL); 
  printf("sum = %d\n", sum);
}

/* The thread will begin control in this function */ 
void *runner(void *param) {

  int i, upper = atoi(param);
  sum = 0;
  for (i = 1; i <= upper; i++) {
    sum += i;
  }
  
  pthread_exit(0);
}
\end{lstlisting}

In this example, both threads are sharing the global variable \texttt{sum}. We have some form of co-ordination here because the parent thread will join the newly-spawned thread (i.e., wait until it is finished) before it tries to print out the value. If it did not join the spawned thread, the parent thread would print out the sum early. This is yet another example of that subject that keeps popping up: co-ordination...




\subsection*{Thread Cancellation}
Thread cancellation is exactly what it sounds like: a running thread will be terminated before it has finished its work. Once the user presses the cancel button on the file upload, we want to stop the upload task that was in progress. The thread that we are going to cancel is called the \textit{target} (because we shoot targets, I guess) and there are two ways a thread might get cancelled~\cite{osc}:

\begin{enumerate}
	\item \textbf{Asynchronous Cancellation:} One thread immediately terminates the target.
	\item \textbf{Deferred Cancellation:} The target is informed that it is cancelled; the target is responsible for checking regularly if it is terminated, allowing it to clean itself up properly. 
\end{enumerate}

For example, in Android, a background task has a function \texttt{isCancelled} that returns a boolean value. If we cancel a task then the \texttt{isCancelled} value returns true, but on its own that does not actually impact the task directly. The task itself is responsible for checking if \texttt{isCancelled} is true and stopping its activity and cleaning up (closing open files, etc.) before it terminates. This is deferred cancellation, and it's possible, though generally poor programming practice, to never check for cancellation.

Given that a thread can effectively ignore a cancellation if it is the deferred cancellation type, why would we ever choose that over asynchronous cancellation? Suppose the thread we are cancelling has some resources. If the thread is terminated in a disorderly fashion, the operating system may not reclaim all resources from that thread. Thus a resource may appear to be in use even though it is not, denying that resource to other threads and processes that may want to use it~\cite{osc}.


The pthread command to cancel a thread is \texttt{pthread\_cancel} and it takes one parameter (the thread identifier). By default, a pthread is set up for deferred cancellation. In the function that runs as a thread, to check if the thread has been cancelled, the function call is \texttt{pthread\_testcancel} which takes no parameters.

Suppose your background task is to upload a bunch of files, consecutively. It is good programming practice to check \texttt{pthread\_testcancel} at the start or end of each iteration of the loop, and if cancellation has been signalled, clean up open files and network connections, and then \texttt{pthread\_exit}. Thus, if the thread has been told to cancel, it will do as it is told within a fairly short period of time.

\bibliographystyle{alphaurl}
\bibliography{252}


\end{document}